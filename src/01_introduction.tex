\cleardoublepage
\chapter{Introduction}
\minitoc\label{sec:introduction}\vspace{.5cm}

\section{Background and Motivation}

Connectivity has become a pivotal driver towards digitalization and automation in industrial environments. 
The evolution of the IoT and Cloud services has facilitated the rise of the fourth industrial generation, Industry 4.0, as a new trend of automation and data exchange in the manufacturing industry.
This new industrial paradigm is characterised by its ability to reconfigure and often optimize autonomously.\\
The fourth industrial revolution Industry 4.0 aims at transforming today’s factories into intelligently connected production information systems that operate well beyond the physical boundaries of the factory premises.
Factories of the future leverage the smart integration of “cyber-physical-systems” and Internet of Things (IoT) solutions in industrial processes\cite{7883994}, moving the industry to next technology while ensuring cyber security.\\
The Fifth Generation (5G) networks will play a key-enabling role in this integration, offering programmable technology platforms able to connect a wide variety of devices in an ubiquitous manner.
The 5G network infrastructure provides highly secure, reliable and resilient cellular connectivity, which is crucial for mission-critical applications.\\
With the increase in the amount of data captured during the manufacturing process, monitoring systems are becoming important factors in decision making for management.\\

Robust determinism and real-time (RT) performance are mandatory for process control and manufacturing systems in modern industrial automation.
Many modern information systems are becoming safety-critical in a general sense because loss of life and property can result from their failure.\cite{1007998}.
The number of computer systems that fall under the safety-critical category is increasing dramatically as computer systems continue to be introduced into many areas that affect our lives.\\
Moreover, there are significant safety assurance challenges that are posed by the reconfigurable and modular nature of smart factories,
which need to quickly adapt to production line changes and minimize downtime due to factory modifications.
This could weaken the confidence in the safety of the factory and result in a reduction of the overall safety case.\\
Overall, there is a growing necessity in providing and assuring a deterministic real-time exchange of information to guarantee safe operations.\\

Network monitoring guides network operators in understanding the current behavior of a network.
Therefore, accurate and efficient monitoring, e.g. collecting up-to-date information about the traffic load, performance
parameters or potential problems is vital to ensure that the network operates according to the intended behavior as well as to troubleshoot any deviations.\\
However, Monitoring multiple network parameters on every link or node without overloading the network or the monitoring system
might be challenging, mainly in large and dense networks.\\

To solve the issue of bandwidth consumption by the network monitoring system, there have been many approaches such as utilizing machine learning models in delay prediction \cite{8638081}.
Another approach proposes a monitoring model based on a subset of nodes in the network and therefore reduce the amounts on collected and processed data \cite{paluchov10}.
Thess approaches do reduce the network overload and provide faster network response in some sense.\\
However, they do not deliver absolute knowledge about the whole network topology due to the
performance optimization trade-offs and therefore providing less reliable results.\\

In this paper, we focus on combining a number of state-of-the-art solutions such as the extended Berkley Packet Filter eBPF \cite{8493077},
Software-defined Networking SDN \cite{6838227} following the latest methods in network monitoring \cite{4644132},
addressing the challenges in monitoring real-time networks \cite{LEE201484}.\\
Some of the features of an ideal real-time network include:
\begin{enumerate}
    \item High Performance: The network must guarantee high throughput and low latency in order to meet real time requirements of complex applications.
    \item Determinism: A network is deterministic when there is little or no jitter during packet transmission.
    \item Fault Tolerance: An important criterion for any safety-critical system is to have high reliability. A safety-critical network must be able to tolerate both permanent and temporary faults without leading to catastrophic results.
    \item Unified Network: A single network is expected to carry different classes of traffic (critical, sub-critical and non-critical traffic) within the same medium.
\end{enumerate}

To help quantify these essential features we propose a real-time network monitoring system, that collects and analyses measurements from the network nodes
while minimizing the impact on normal traffic and ongoing network operations in order to maintain the real-time network features.\\
The extended Berkeley Packet Filter eBPF can significantly reduce the amount of traffic generated in the data collection phase
as well as the processing power needed by extending the access to kernel-level functions.\\
The system collects measurement data passively as well as actively depending on the requirements.\\
A more detailed overview will follow in \ref{sec:sota}.\\

The traditional network monitoring systems deliver relative results that are sufficient to the regular user.
However, these tools aren't suitable for IIot applications because they arent't built to meet industrial real-time network requirements.
We exploit the features of eBPF \cite{8850758}, such as the integration with the Linux eXpress Data Path (XDP) for early access to incoming network packets,
to gain more accurate measurements from the network while minimizing the traffic and processing load. More details about eBPF are found in section \ref{sec:contrib1}.\\
An ideal approach needs to find balance between monitoring all necessary details in the whole network while keeping the monitoring
overhead low without any compromizes that could lead to unexpected network behavior. This is mainly crucial in safety-critical fields.

\section{Assumptions and Scope}
Future industrial communications involve high data rate best effort traffic working alongside real-time heterogeneous traffic
for time-critical applications with hard deadlines.\\
This work addresses some of the challenges regarding monitoring networks deployed in time-critical fields.\\
The structure of the following sections is as follows; We present existing monitoring challenges and solutions for traditional networks
and discuss eBPF technology as well as the related state-of-the-art network technologies in section \ref{sec:sota}.
The third \ref{sec:reqs} and fourth \ref{sec:contrib1} sections are dedicated to discussing the requirements, the proposed
solution and used methodology. The evaluation, testing results and the conclusions of the work follow after.\\
